%!Tex Root = ../main.tex
% ./Packete.tex
% ./Design.tex
% ./Deklarationen.tex
% ./Aufgabe1.tex
% ./Aufgabe2.tex
% ./Aufgabe3.tex
% ./Aufgabe4.tex
% ./Appendix.tex

\if\juergen1{
\section{Organisatorisches}

\begin{frame}{Organisatorisches}{Abgaben}
  \begin{itemize}
    \item schreibt euren Namen und Matrikelnummer auf die Abgaben
    \item \alert{Vorrechnen} nicht vergessen:
      \begin{enumerate}
        \item entweder generell \alert{immer mal wieder Melden}, dann zählt das irgendwann als Vorrechnen
        \item oder \alert{vor dem Tutorat ansprechen}, dann werdet ihr während des Tutorats dazu gefragt, ob ihr zu einer Aufgabe vielleicht irgendetwas \alert{gehaltvolles} sagen könnt
      \end{enumerate}
    % \item macht euch keinen Kopf, wenn ihr wenige Punkte auf einzelnen Blättern habt. Die Blätter haben unterschiedlich viele Punkte, die man maximal erreichen kann und sind daher untereinander sowieso nicht vergleichbar. Davon abgesehen ist die Punktzahl erstmal sowieso irrelevant für die Studienleistung, sondern der Prozentsatz sinnvoll bearbeiteter Aufgaben.
  \end{itemize}
\end{frame}
}\fi
