%!Tex Root = ../main.tex
% ./Packete.tex
% ./Design.tex
% ./Deklarationen.tex
% ./Vorbereitung.tex
% ./Aufgabe1.tex
% ./Aufgabe2.tex
% ./Aufgabe3.tex
% ./Aufgabe4.tex

\section{Appendix}

\begin{frame}[allowframebreaks]{Appendix}{Hinweis zu Hypercubes und Karnaugh Map}
  \begin{itemize}
    \item Wenn sich zwei Knoten nur durch genau $1$ Bit unterscheiden, dann werden sie durch eine Kante verbunden.
  \end{itemize}
  \begin{figure}
      \includegraphics[width=0.8\textwidth, center]{./figures/hypercubes.png}
      \caption{Hypercubes von $n=0$ bis $n=4$}
  \end{figure}
  \newpage
  \begin{figure}
    \centering
    \begin{subfigure}{0.4\textwidth}
       \includegraphics[height=0.4\textheight, center]{./figures/donut.png}
       \caption{Torus oder umgangssprachlich Donut}
    \end{subfigure}
    \begin{subfigure}{0.4\textwidth}
       \includegraphics[height=0.4\textheight, center]{./figures/K_map.png}
       \caption{Beispiel für Karnaugh Map}
    \end{subfigure}
  \end{figure}
  \begin{itemize}
      \item Anordnung ist so, weil man einen Hypercube bis $n=4$ (Tesseract) in einer flachen Tabelle darstellen will und im Hypercube gibt es immer nur eine Kante, wenn es nur genau $1$ Bit Unterschied bei $2$ Knoten gibt.
      \item In der Karnaugh Map sind entsprechend immer nur Bitvektoren benachbart, die sich nur in einem Bit unterscheiden. 
      \item Wie bei einem Donut sind Reihen und Spalten miteinander von unten nach oben und von links nach rechts über die Tabelle hinausgehend benachbart.
      \item Von $\overline{a}b$ zu $ab$ gibt es nur $1$ Änderung. Von $\overline{a}b$ zu $a\overline{b}$ gibt es dagegen $2$ Änderungen. Daher sind $\overline{a}b$ und $ab$ benachbart.
  \end{itemize} 
\end{frame}
 
% in welcher Reihenfolge am sinnvollsten bearbeiten
% Fehler vom letzten Tutorat korrigiert

% wie Kurschluss
% Beweis
% diese allgemeine mit Wertigkeit usw.
% Wie Transistoren funktionieren
% wie Hypercubes genau funktionieren

% bei KNF und Karnaugh Map muss man immer von der negierten Maxtermen ausgehen bei eintragen in die Tabelle. Wobei es einfach ausgedrückt einfach die Invertierung der anderen Tabelle ist
% wie man sich P-Kanal und N-Kanal gut merken kann, p und 0, N und I und lässt genau dann durch, wenn es 0 ist
