%!Tex Root = ../main.tex
% ./Packete.tex
% ./Design.tex
% ./Deklarationen.tex
% ./Vorbereitung.tex
% ./Aufgabe1.tex
% ./Aufgabe2.tex
% ./Aufgabe4.tex
% ./Appendix.tex

\section{Aufgabe 3}

\setcounter{exercise}{1}

\begin{frame}[allowframebreaks]{Aufgabe \thesection}{NOR RS-Flipflop}
  \begin{solution}
    \begin{itemize}
      \item Es gibt \alert{fünf stabile Belegungen}:
      \begin{enumerate}
        \item $a = 0$, $b = 0$, $c = 0$, $d = 1$
        \item $a = 0$, $b = 0$, $c = 1$, $d = 0$
        \item $a = 0$, $b = 1$, $c = 1$, $d = 0$
        \item $a = 1$, $b = 0$, $c = 0$, $d = 1$
        \item $a = 1$, $b = 1$, $c = 0$, $d = 0$
      \end{enumerate}
    \end{itemize}
  \end{solution}
  \begin{solution}
    \begin{itemize}
      \item bei a = b = 0 wird der aktuelle \alert{Wert gehalten}
      \item bei a = 0, b = 1 wird c auf 1 und d auf 0 gesetzt 
      \item bei a = 1, b = 0 wird c auf 0 und d auf 1 gesetzt
    \end{itemize}
  \end{solution}
  \begin{solution}
    \begin{itemize}
      \item a und b sind \alert{active-high}, da sie durch das Heben auf 1 aktiviert werden
    \end{itemize}
  \end{solution}
  \begin{solution}
    \begin{itemize}
      \item die Belegung a = 1, b = 1 ergibt keinen Sinn, da 
      \begin{itemize}
        \item es bei gleichzeitigem Absenken von a und b zu \alert{Flackern} kommen kann
        \item da es für die diese Eingangsbelegung \alert{nur einen stabilen Zustand} gibt. Daher kann \alert{nur ein Wert \enquote{gespeichert}} werden
      \end{itemize}
    \end{itemize}
  \end{solution}
\end{frame}
