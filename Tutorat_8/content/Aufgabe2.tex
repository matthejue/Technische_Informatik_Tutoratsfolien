%!Tex Root = ../main.tex
% ./Packete.tex
% ./Design.tex
% ./Deklarationen.tex
% ./Vorbereitung.tex
% ./Aufgabe1.tex
% ./Aufgabe3.tex
% ./Aufgabe4.tex
% ./Appendix.tex

\section{Aufgabe 2}

\setcounter{exercise}{1}

\begin{frame}[allowframebreaks]{Aufgabe \thesection}{Methode von Petrick}
  \begin{requirementsnoinc}
    \begin{itemize}
      \item \alert{Absorption:} $A+AB=A$
      \item \alert{Korollar:} $AA=A$
    \end{itemize}
  \end{requirementsnoinc}
  \begin{exercisenoinc}
    \begin{table}
      \centering
      \begin{tblr}{
          vlines, hlines,
          cells={c, white},
          row{1} = {PrimaryColor, fg=white},
          column{1} = {PrimaryColor, fg=white},
        }
                 & a & b & c & d & e & f \\
      A & & 1 & & 1 & & \\  
      B & 1 & & 1 & 1 & & \\
      C & & & 1 & 1 & & 1 \\
      D & & 1 & & & 1 & \\
      E & & & 1 & 1 & 1 & \\
      F & 1 & 1 & & & & 1
      \end{tblr}
    \end{table}

  \begin{itemize}
    \item \alert{Minterme:} $\{a,b,c,d,e,f\}$
    \item \alert{Primimplikanten:} $\{A,B,C,D,E,F\}$
  \end{itemize}
\end{exercisenoinc}

% \begin{solutionnoinc}
%   \begin{table}
%     \centering
%     \begin{tblr}{
%         vlines, hlines,
%         cells={c, white},
%         row{1} = {PrimaryColor, fg=white},
%         column{1} = {PrimaryColor, fg=white},
%       }
%
%                  & a & b & c & d & e & f \\
%     A & & 1 & & 1 & & \\
%     B & 1 & & 1 & 1 & & \\
%     C & & & 1 & 1 & & 1 \\
%     D & & 1 & & & 1 & \\
%     E & & & 1 & 1 & 1 & \\
%     F & 1 & 1 & & & & 1
%     \end{tblr}
%   \end{table}
%   \begin{itemize}
%     % \item keine wesentlichen Primimplikanten $\Rightarrow$ wir können die erste Reduktionsregel nicht anwenden
%     % \item Minterm d dominiert Minterm c $\Rightarrow$ zweite Regel: Lösche Spalte d
%   \end{itemize}
% \end{solutionnoinc}

\begin{solutionnoinc}
  \begin{table}
    \centering
    \begin{tblr}{
        vlines, hlines,
        cells={c, white},
        row{1} = {PrimaryColor, fg=white},
        column{1} = {PrimaryColor, fg=white},
      }
                 & a & b & c & e & f \\
    A & & 1 & & & \\ 
    B & 1 & & 1 & & \\
    C & & & 1 & & 1 \\
    D & & 1 & & 1 & \\
    E & & & 1 & 1 & \\
    F & 1 & 1 & & & 1
    \end{tblr}
  \end{table}
  \begin{itemize}
    \item \alert{Regel 2:} d dominiert c $\Rightarrow$ Lösche d.
    % \item keine wesentlichen Primimplikanten $\Rightarrow$ wir können die erste Reduktionsregel nicht anwenden
    % \item wir können auch die zweite Regel nicht anwenden
    % \item Die Primimplikanten D und F dominieren den Primimplikanten A $\Rightarrow$ dritte Regel: Lösche Zeile A
  \end{itemize}  
\end{solutionnoinc}

\begin{solutionnoinc}
  \begin{table}
    \centering
    \begin{tblr}{
        vlines, hlines,
        cells={c, white},
        row{1} = {PrimaryColor, fg=white},
        column{1} = {PrimaryColor, fg=white},
      }
                 & a & b & c & e & f \\
    B & 1 & & 1 & & \\
    C & & & 1 & & 1 \\
    D & & 1 & & 1 & \\
    E & & & 1 & 1 & \\
    F & 1 & 1 & & & 1
    \end{tblr}
  \end{table}
  \begin{itemize}
    \item \alert{Regel 3:} D (oder F) dominiert A $\Rightarrow$ Lösche A
    % \item keine wesentlichen Primimplikanten $\Rightarrow$ erste Reduktionsregel nicht anwendbar
    % \item zweite Regel nicht anwendbar
    % \item dritte Regel nicht anwendbar
    % \item $\Rightarrow$ zyklisches Überdeckungsproblem
    % \item $\Rightarrow$ Methode von Petrick
  \end{itemize} 
\end{solutionnoinc}

\begin{solutionnoinc}
    \begin{table}
      \centering
      \begin{tblr}{
          vlines, hlines,
          cells={c, white},
          row{1} = {PrimaryColor, fg=white},
          column{1} = {PrimaryColor, fg=white},
        }
                  & a & b & c & e & f \\
      B & 1 & & 1 & & \\
      C & & & 1 & & 1 \\
      D & & 1 & & 1 & \\
      E & & & 1 & 1 & \\
      F & 1 & 1 & & & 1
      \end{tblr}
    \end{table}
  \begin{itemize}
    \item Keine weitere Reduktionsregel mehr anwendbar $\Rightarrow$ Petrick
  \end{itemize} 
\end{solutionnoinc}
\begin{solutionnoinc}
  \begin{itemize}
    \item \alert{Petrick:} \tiny$\begin{aligned}[t]
        &\underbrace{(B+F)}_{\text{überdecken a}}\underbrace{(D+F)}_{\text{überdecken b}}\underbrace{(B+C+E)}_{\text{usw.}}(D+E)(C+F)\\
        &=(BD + BF + FD + F) \cdot (BD + BE + CD + CE + ED + E) \cdot (C + F)\\
        &=(BD + BF + FD + F) \cdot (BDC + BEC + CD + CE + EDC + BDF + BEF + CDF + CEF + EDF + EF)\\
        &=(BD + BF + FD + F) \cdot (CD + CE + BDF + EF)\\
        &=BDC + BDCE + BDF + BDEF + BFCD + BFCE + BFD + BFE + FDC + FDCE + FDB + FDE + FCD + FCE + FBD + FE\\
        &=BDC + BDF + BFD + FDC + FBD + \underline{FE}\\
        % &=\underline{EF} + DEF + CEF + BDF + BCD + CDF + BEF + CDEF + BDEF + BCEF + BCDF + BCDE + BCDEF
      \end{aligned}$
    \item Produkt mit den wenigsten Termen auswählen. Wenn es zwei solche Produkte gibt, dann wählt man die Terme mit den wenigsten Literalen
    \item Minimalpolynom $f_{min} = F + E$
  \end{itemize}
\end{solutionnoinc}
\end{frame}
