%!Tex Root = ../main.tex
% ./Packete.tex
% ./Design.tex
% ./Deklarationen.tex
% ./Vorbereitung.tex
% ./Aufgabe1.tex
% ./Aufgabe2.tex
% ./Aufgabe4.tex
% ./Appendix.tex

\section{Aufgabe 3}

\setcounter{exercise}{1}

\begin{frame}[allowframebreaks]{Aufgabe \thesection}{Darstellung von Festkommazahlen - Dezimalsystem}
  \begin{solution}
    \begin{itemize}
      \item $x=10^n-1$, $\displaystyle [d_{n}d_{n-1}\ldots d_{0}]_9=\sum_{i=0}^{n-1}d_{i}\cdot 10^{i}-d_{n}\cdot\ x$
      \item Die \alert{größte darstellbare Zahl} ist für beliebige $n>0$: $10^n - 1$ und für $n=4$: $10^4-1 = 9999$
      \item Für einen symmetrischen Zahlenbereich kann man sich erstmal überlegen, dass man zwei Darstellung für die $0$ braucht und auch die Negation der größten darstellbaren Zahl darstellbar sein muss. D.h. für z.B. $n = 4$ muss man auch $-9999$ darstellen können. Und diesen Grenzfall erreicht man, indem man von $0$ $9999$ abzieht.
    \end{itemize}
  \end{solution}
  \begin{exercisenoinc}
    \begin{itemize}
      \item \alert{Komplementieren} der Ziffern in der \alert{Neuner-KomplementDarstellung} definiert werden muss, damit das folgende Lemma gilt:
      \item \alert{Lemma:} Sei $a$ eine Festkommazahl im Dezimalsystem, $a'$ die Festkommazahl im Dezimalsystem, die aus $a$ durch Komplementieren aller Ziffern hervorgeht. Dann gilt $[a']_9 + [a]_9 = 0$.
    \end{itemize}
  \end{exercisenoinc}
  \begin{solutionnoinc}
    \begin{itemize}
      \item Das Einerkomplent im Binärsystem interpretiert Zahlen mit führenden $0$en als nichtnegative Zahlen und Zahlen
mit führenden $1$en als negative Zahlen. Um das Einerkomplement einer Zahl zu bekommen werden alle Bits
invertiert.
      \item Analog dazu kann man das Neunerkomplement im Dezimalsystem bilden. In der Aufgabenstellung ist definiert, dass auch hier die höchstwertigste Ziffer nur $0$ oder $1$ sein kann. 
      \item Das Invertieren wäre hier die Ersetzung nach folgender Funktion:\\[0.25cm] $inv : \{0, 1, \ldots, 8, 9\} \Rightarrow \{0, 1, \ldots, 8, 9\} : d_i \mapsto \begin{dcases}
          9-d_i & 0\le i < n \\
          1-d_i & i = n\;und\;d_i\in\{0, 1\}
        \end{dcases}
        $
      \item Für $01784$ ist die komplementierte Zahl: $18215$\\[0.25cm]
        $[01784]_{0}+[18215]_{9}=1784+(8215-9999)=(1784+8215)-9999=9999-9999=0$
    \end{itemize}
  \end{solutionnoinc}
\end{frame}
