%!Tex Root = ../main.tex
% ./Packete.tex
% ./Design.tex
% ./Deklarationen.tex
% ./Vorbereitung.tex
% ./Aufgabe1.tex
% ./Aufgabe2.tex
% ./Aufgabe3.tex
% ./Aufgabe4.tex

\section{Appendix}

\begin{frame}[allowframebreaks]{Appendix}{Kleine Korollare\vspace{0.5cm}}
  \begin{align*}
    x + 1 &= (x + 1) \cdot 1 & \text{Neutrales Element der Konjunktion}\\
    &= (x + 1) \cdot (x + \neg x) & \text{Komplement}\\
    &= x + (1 \cdot \neg x)& \text{Distributivität}\\
    &= x + \neg x& \text{Neutrale Element der Konjunktion}\\
    &=1& \text{Komplement}\\[0.5cm]
  \end{align*}
  \begin{align*}
    x \cdot 0 &= (x \cdot 0) + 0 & \text{Neutrales Element der Disjunktion}\\
              &= (x \cdot 0) + (x \cdot \neg x) & \text{Komplement}\\
              &= x \cdot (0 + \neg x) & \text{Distributivität}\\
              &= x \cdot \neg x & \text{Neutrale Element der Disjunktion}\\
              &=0 & \text{Komplement}
  \end{align*}
  \pagebreak
  \begin{align*}
    y \cdot  y &= y\cdot y + 0             & \text{Neutrales Element}\\
               &= y\cdot y + y\cdot \neg y & \text{Komplement}\\
               &= (y+\neg y) \cdot  y      & \text{Distributivität}\\
               &= 1 \cdot  y               & \text{Komplement}\\
               &= y                        & \text{Neutrales Element}
  \end{align*}
\end{frame}
