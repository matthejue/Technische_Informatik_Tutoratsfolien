%!Tex Root = ../main.tex
% ./Packete.tex
% ./Design.tex
% ./Deklarationen.tex
% ./Vorbereitung.tex
% ./Aufgabe2.tex
% ./Aufgabe3.tex
% ./Aufgabe4.tex
% ./Appendix.tex

\section{Aufgabe 1}

\setcounter{exercise}{1}

\begin{frame}[allowframebreaks]{Aufgabe \thesection}{Eindeutigkeit des Komplements, Boolesche Algebra}
  \begin{solutionnoinc}
    \begin{itemize}
      \item Existenz und Eindeutigkeit neutraler Elemente bereits in Vorlesung gezeigt $\rightarrow$ Es kann $1 = x + \neg x$ verwendet werden
      \item \alert{Annahme 1:} $x + y = 1$
      \item \alert{Annahme 2:} $x · y = 0$
      \item \alert{Analoges Korollar zur Vorlesung:} $1 = x + \neg x$
    \end{itemize}
  \end{solutionnoinc}
  \begin{solution}
    \begin{align*}
      y&=y\cdot1&{({\text{neutrales~Element)}}}\\
      &{=y\cdot(x+\neg x)}&{(\text{Korollar})}\\
      &{=(y\cdot x)+(y\cdot \neg x)}&{(\text{Distributivität})}\\
      &{=(x\cdot y)+(y\cdot \neg x)}&{(\text{Kommutativität})}\\
      &{=0+(y\cdot \neg x)}&{(\text{Annahme 2})}\\
      &{=(x\cdot\neg x)+(y\cdot\neg x)}&{(\text{Korollar})}\\
      &{=\neg x\cdot (x+ y)}&{(\text{Kommutativität + Distributivität})}\\
      &{=\neg x\cdot 1}&{(\text{Annahme 1})}\\
      &=\neg x &(\text{neutrales~Element})
    \end{align*}
  \end{solution}
  \begin{solutionnoinc}
    \resizebox{\textwidth}{!}{
      \begin{minipage}[t]{8cm}
        \begin{align*}
            {{(x\cdot y)+(\neg x\cdot z)}} &{{=(x\cdot y)+((x\cdot y)\cdot z)+((\neg x\cdot z)+((\neg x\cdot z)\cdot y)}}&{{(\text{Absorption})}}\\
                                        &{{=(x\cdot y)+(\neg x\cdot z)+((x\cdot y)\cdot z)+((\neg x\cdot z)\cdot y)}}&{{(\text{Kommutativität})}}\\
                                        &{{=(x\cdot y)+(\neg x\cdot z)+(x \cdot (y\cdot z))+(\neg x\cdot(z\cdot y))}}&{{(\text{Assoziativität})}}\\
                                        &{{=(x\cdot y)+(\neg x\cdot z)+(x \cdot (y\cdot z))+(\neg x\cdot(y\cdot z))}}&{{(\text{Kommutativität})}}\\
                                        &{{=(x\cdot y)+(\neg x\cdot z)+((x+ \neg x)\cdot(y\cdot z))}}&{{(\text{Kommutativität}+\text{Distributivität})}}\\
                                        &{{=(x\cdot y)+(\neg x\cdot z)+(y\cdot z)}}&{{(\text{Kommutativität}+\text{Komplement})}}
        \end{align*}
      \end{minipage}
    }
  \end{solutionnoinc}
  \begin{Sidenote}
    \begin{itemize}
      \item Der zweite Teil geht analog durch Verweisen auf das Dualitätsprinzip oder durchrechnen mit genau derselben Axiomanwendung in genau derselben Reihenfolge
    \end{itemize}
  \end{Sidenote}
\end{frame}
