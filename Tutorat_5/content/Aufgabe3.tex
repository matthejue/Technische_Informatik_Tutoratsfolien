%!Tex Root = ../main.tex
% ./Packete.tex
% ./Design.tex
% ./Deklarationen.tex
% ./Vorbereitung.tex
% ./Aufgabe1.tex
% ./Aufgabe2.tex
% ./Aufgabe4.tex
% ./Appendix.tex

\section{Aufgabe 3}

\setcounter{exercise}{1}

\begin{frame}[allowframebreaks]{Aufgabe \thesection}{Boolesche Algebra}
  \begin{solutionnoinc}
    \tiny
    \begin{table}
      \centering
      \begin{tblr}{
        cells = {white, c},
        row{1} = {PrimaryColor,fg=white},
        vline{3} = {-}{},
        vline{5} = {-}{},
        vline{8} = {-}{},
      }
      x & y & $\lnot x$ & $\lnot y$ & $\lnot x \lor \lnot y$ & $x \land y$ & $\lnot (x \land y)$ & $\lnot x \land \lnot y$ & $x \lor y$ & $\lnot (x \lor y)$ \\\hline
      0 & 0 & 1 & 1 & 1 & 0 & 1 &                                                                  1 & 0 & 1  \\
      0 & 1 & 1 & 0 & 1 & 0 & 1 &                                                                  0 & 1 & 0  \\
      1 & 0 & 0 & 1 & 1 & 0 & 1 &                                                                  0 & 1 & 0 \\
      1 & 1 & 0 & 0 & 0 & 1 & 0 &                                                                  0 & 1 & 0 
      \end{tblr}
    \end{table}
  \end{solutionnoinc}
% die beste all-in-one-Lösung für Tabellen ist: https://packages.oth-regensburg.de/ctan/macros/latex/contrib/tabularray/tabularray.pdf (als Empfehlung)
% merci
% in diesem Dokument bei Aufgabe 1 ist nen Beispiel wie man Tabularray verwendet, falls du es verwenden willst. Wenn man einmal zu Tabularray wechselt sind alle Sorgen, die man sonst mit Tabellen hat vergessen.
\end{frame}
