%!Tex Root = ../main.tex
% ./Packete.tex
% ./Design.tex
% ./Deklarationen.tex
% ./Vorbereitung.tex
% ./Aufgabe1.tex
% ./Aufgabe2.tex
% ./Aufgabe4.tex
% ./Bonus.tex

\section{Aufgabe 3}

\setcounter{exercise}{1}

\begin{frame}[allowframebreaks]{Exercise \thesection}{Darstellung von Festkommazahlen - Dezimalsystem}
  \begin{solution}
    \begin{itemize}
      \item $x=10^n-1$
      \item Die \alert{größte darstellbare Zahl} ist $10^n - 1$
      \item Für einen symmetrischen Zahlenbereich kann man sich erstmal überlegen, dass man zwei Darstellung für die $0$ braucht und auch die Negation der größte darstellbare Zahl darstellbar sein muss. Dh für zB $n = 4$ muss man auch $-9999$ darstellen können. Und diesen Grenzfall erreicht man, indem man von $0$ $9999$ abzieht.
    \end{itemize}
  \end{solution}
\end{frame}
