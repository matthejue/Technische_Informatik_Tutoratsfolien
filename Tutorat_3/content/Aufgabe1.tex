%!Tex Root = ../main.tex
% ./Packete.tex
% ./Design.tex
% ./Deklarationen.tex
% ./Vorbereitung.tex
% ./Aufgabe2.tex
% ./Aufgabe3.tex
% ./Aufgabe4.tex
% ./Appendix.tex

\section{Aufgabe 1}

\setcounter{exercise}{1}

\begin{frame}[allowframebreaks]{Aufgabe \thesection}{Zahlensysteme und Darstellung negativer Kommazahlen}
  \if\gabriella1{
  \begin{requirementsnoinc}
    \begin{itemize}
      \item Der Wert von $[d_n,d_{n-1},...,d_0]$ entspricht:
      \begin{itemize}
          \item $[d]_{BV} = \sum\limits^{n-1}_{i=0}d_i 2^i \cdot(-1)^{d_n}$
          \item $[d]_1 = \sum\limits^{n-1}_{i=0}d_i 2^i - (d_n 2^n -1)$
          \item $[d]_2 = \sum\limits^{n-1}_{i=0}d_i 2^i - d_n 2^n$
      \end{itemize}
    \end{itemize}
  \end{requirementsnoinc}
  \begin{exercisenoinc}
    \begin{itemize}
      \item Konvertiere folgende Zahlen mit möglichst wenig Bits in B\&V, 1er- und 2er-Komplement Darstellung:
      \begin{itemize}
          \item $2342_{10}$
          \item $-BFDC_{16}$
          \item $-10_{16}$
          \item $255_8$
      \end{itemize}
    \end{itemize}
  \end{exercisenoinc}
  Mögliche Vorgehensweise
  \begin{block}{Zuallererst}
        \begin{itemize}
            \item berechne Binärzahl (ohne Vorzeichen)
        \end{itemize}
        \end{block}
        \begin{block}{Positives Vorzeichen}
            \begin{itemize}
                \item BV: setze eine 0 davor
                \item 1er-Kompl.: gleiches wie bei BV
                \item 2er-Kompl.: gleiches wie bei BV
            \end{itemize}
        \end{block}
        \begin{block}{Negatives Vorzeichen}
            \begin{itemize}
                \item BV: setze eine 1 davor
                \item 1er-Kompl.: invertiere die Binärzahl und setze eine 1 davor
                \item 2er-Kompl.: nimm das 1er-Komplement und addiere 1 dazu
            \end{itemize}
        \end{block}
    }\fi
  \begin{solution}
    Aufgabe 1a)\newline\newline
    $2342_{10} = 100100100110_2 \newline\newline = [0100100100110]_{BV} \newline\newline = [0100100100110]_{1} \newline\newline = [0100100100110]_{2}$
  \end{solution}
  \begin{solution}
    Aufgabe 1b)\newline\newline
    $–BFCD_{16} = - 1011 1111 1100 1101_2 \newline\newline = [11011 1111 1100 1101]_{BV} \newline\newline = [1 0100 0000 0011 0010]_1 \newline\newline = [1 0100 0000 0011 0011]_2$
  \end{solution}
  \begin{solution}
    Aufgabe 1c)\newline\newline
    $–10_{16} = - 1 0000_2 \newline\newline = [11 0000]_{BV} \newline\newline = [10 1111]_1 \newline\newline = [11 0000]_2$  
  \end{solution}
  \begin{solutionnoinc}
    Aufgabe 1d)\newline\newline
    $255_{8} = 10 101 101_2 \newline\newline = [010 101 101]_{BV} \newline\newline = [010 101 101]_{1} \newline\newline = [010 101 101]_{2}$
  \end{solutionnoinc}
\end{frame}
