%!Tex Root = ../main.tex
% ./Packete.tex
% ./Design.tex
% ./Deklarationen.tex
% ./Aufgabe1.tex
% ./Aufgabe2.tex
% ./Aufgabe3.tex
% ./Aufgabe4.tex
% ./Appendix.tex

\if\juergen1{
\section{Organisatorisches}

\begin{frame}{Organisatorisches}{Abgaben}
  \begin{itemize}
    \item schreibt \alert{bitte} euren \alert{Namen} und \alert{Matrikelnummer} auf die Abgaben
    \item \alert{Vorrechnen} nicht vergessen:
      \begin{enumerate}
        \item entweder generell \alert{immer mal wieder Melden}, dann zählt das irgendwann als Vorrechnen
        \item oder \alert{vor dem Tutorat ansprechen}, dann werdet ihr während des Tutorats dazu gefragt, ob ihr zu einer Aufgabe vielleicht irgendetwas \alert{gehaltvolles} sagen könnt
      \end{enumerate}
    \item um die Leute, die in die Tutorate kommen zu \alert{belohnen} und dem Ereignis entgegenzuwirken, dass das Tutorat irgendwann \alert{leer} ist, werden die Folien während des Tutorats auf einem \alert{USB-Stick} verteilt
      \begin{itemize}
        \item die Folien \alert{aller bisherigen Tutorate} werden \alert{immer} auch alle auf dem USB-Stick sein
      \end{itemize}
  \end{itemize}
\end{frame}
}\fi
