%!Tex Root = ../main.tex
% ./Packete.tex
% ./Design.tex
% ./Deklarationen.tex
% ./Vorbereitung.tex
% ./Aufgabe1.tex
% ./Aufgabe2.tex
% ./Aufgabe4.tex
% ./Appendix.tex

\section{Aufgabe 3}

\setcounter{exercise}{1}

% \begin{frame}[allowframebreaks]{Aufgabe \thesection}{Zustandsdiagramme, Mealy-Automaten}
%
% \end{frame}

    % \begin{frame}{Aufgabe 3}{}
    %     \begin{block}{Aufgabe}
    %         Zeichne Zustandsdiagramm von:\\
    %         Incrementer einer Binärzahl als Mealy-Atomat:
    %         \begin{itemize}
    %             \item niedrigstwertiges Bit wird zuerst gelesen
    %             \item auf das höchstwertige Bit folgt $\#\#$
    %             \item das erste $\#$ wird durch das Überlauf-Bit ersetzt, das zweite bleibt stehen
    %             \item Symbole nach der Endmarkierung sollen durch $\#$ ersetzt werden
    %         \end{itemize}
    %     \end{block}
    % \end{frame}

    \begin{frame}{Aufgabe \thesection}{Zustandsdiagramme, Mealy-Automaten}
        \begin{solutionnoinc}
            \includegraphics[height=0.5\paperheight, center]{./figures/Mealy-Increment.png}
        \end{solutionnoinc}
    \end{frame}

% Vorgehen: Ein Zustand gibt an, wie weit man noch vor einem möglichen fertigen Wort entfernt ist. Man fragt sich bei jedem Zustand für alle Symbole aus Σ Σ, was es jetzt für die Entfernung zu einem fertigen Wort bedeuten würde. Z.B. bei 2 Zustände vor dem Endzustand und alle Zeichen durchgehen und nachdenken, was würde es jetzt bedeuten wenn jetzt dieses Zeichen kommt? Zu welchem Zustand würde man zurückgeworfen werden? Nicht vergessen, dass es auch Schleifen gibt! Veilleicht von den Endzuständen aus anfangen und wenn man weiß, dass es mit Sufix 110 endet diese schonmal hinschreiben.
