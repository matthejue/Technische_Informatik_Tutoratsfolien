%!Tex Root = ../main.tex
% ./Packete.tex
% ./Design.tex
% ./Deklarationen.tex
% ./Vorbereitung.tex
% ./Aufgabe1.tex
% ./Aufgabe2.tex
% ./Aufgabe4.tex
% ./Bonus.tex

\section{Aufgabe 3}

\setcounter{exercise}{1}

\begin{frame}[allowframebreaks]{Aufgabe \thesection}{XYZ}
  \begin{requirementsnoinc}
    \[
      \sum_{j=1}^{m}p(a_{j})=1,p(a_{i})> 0.5
    \]
  \end{requirementsnoinc}
  \begin{solutionnoinc}
    \begin{align*}
      &\sum_{j=1}^{i-1}p(a_{j})+p(a_{i})+\sum_{j=i+1}^{m}p(a_{j})=1\\
      &\sum_{j=1}^{i-1}p(a_{j})+\sum_{j=i+1}^{m}p(a_{j})< 0.5
    \end{align*}
  \end{solutionnoinc}
  \begin{solutionnoinc}
    \begin{itemize}
      \item Beim Bau des binären Baums sind alle Häufigkeitsteilsummen kleiner als $p(a_i)$, daher wird $a_i$ erst zu dem Baum hinzugefügt, nachdem alle andere Knoten bereits zu einem Knoten zusammengefügt wurden. 
      \item Dieser wird durch eine Kante mit einem neuen Knoten (Wurzel) verbunden, die andere Kante der Wurzel führt direkt zu $a_i$.
    \end{itemize}
    \vspace{0.5cm}
    \resizebox{\textwidth}{!}{
      \begin{minipage}[t]{16cm}
        \ctikzfig{3a}
      \end{minipage}
    }
  \end{solutionnoinc}
\end{frame}

\begin{frame}[allowframebreaks]{Aufgabe \thesection}{XYZ}
  \begin{requirementsnoinc}
    \begin{itemize}
      \item $p(a_{i})< \frac{1}{3}$ und $|c(a_{i})|=1$
    \end{itemize}
  \end{requirementsnoinc}
  \begin{solutionnoinc}
    \begin{itemize}
      \item Wegen $|c(ai )| = 1$ ist $a_i$ einer der beiden direkten Vorgänger der Wurzel des Codebaumes. Für den anderen Vorgänger $v$ gilt:
      \begin{enumerate}[a)]
        \item Er ist mit einer Häufigkeitssumme $> 2/3$ beschriftet, da die Summe an der Wurzel $1$ ist und $p(a_i) < 1/3$.
        \item Der Vorgänger $v$ ist kein Blatt wegen $m \ge 3$.
      \end{enumerate}
      \item Also hat $v$ mindestens einen Vorgänger $v_1$ mit Häufigkeitssumme größer als $\frac{1}{3}$. Der Algorithmus hätte dann aber zum Zeitpunkt des Einfügens von $v$ nicht $v_1$, sondern einen Knoten mit einer kleineren Häufigkeitssumme als Vorgänger von $v$ auswählen müssen. Ein solcher Kandidat wäre z.B. $a_i$ mit $p(a_i) < 1/3$ gewesen.
      \item Widerspruch!
    \end{itemize}
  \end{solutionnoinc}
  \begin{solutionnoinc}
    \ctikzfig{3b}
    % \resizebox{\textwidth}{!}{
    % \begin{minipage}[t]{16cm}
    %   \end{minipage}
    % }
  \end{solutionnoinc}
\end{frame}
