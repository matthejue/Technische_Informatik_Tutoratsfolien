%!Tex Root = ../main.tex
% ./Packete.tex
% ./Design.tex
% ./Deklarationen.tex
% ./Vorbereitung.tex
% ./Aufgabe1.tex
% ./Aufgabe3.tex
% ./Aufgabe4.tex
% ./Appendix.tex

\section{Aufgabe 2}

\setcounter{exercise}{1}

\begin{frame}[allowframebreaks, t]{Aufgabe \thesection}{ON-Menge und Literale}
  \begin{columns}
    \begin{column}{0.5\textwidth}
      \begin{itemize}
        \item \alert{Behauptung:} $m \leq m' \Rightarrow L(m') \subseteq L(m)$
        \begin{itemize}
          \item sei $L: BE(X_n) \rightarrow \mathcal{P}(\{\overline{s} \mid s\in X_n\}\cup X_n)$
        \end{itemize}
      \end{itemize}
    \end{column}
    \begin{column}{0.5\textwidth}
      \begin{figure}[H]
        \resizebox{\textwidth}{!}{
          \begin{minipage}[t]{16cm}
            \ctikzfig{2_hypercubes}
          \end{minipage}
        }
        \caption{Veranschaulichung anhand eines Beispiels}
      \end{figure}
    \end{column}
  \end{columns}
    \begin{columns}
      \begin{column}[t]{0.5\textwidth}
        \resizebox{\textwidth}{!}{
          \begin{minipage}[t]{8cm}
            \begin{itemize}
              \item \alert{Beweis durch Kontraposition:}
              \begin{itemize}
                \item $\mathrm{Z\kern-.3em\raise-0.5ex\hbox{Z}}$: $L(m') \not\subseteq L(m) \Rightarrow ON(m) \not\subseteq ON(m')$
                \item \alert{es gilt:} $L(m) \subset L(m')$, wobei $m$ ein Monom ist und $m' = mx_i^{\omega_i}$
                \item man betrachte $(\omega_1,\ldots, \omega_i, \ldots \omega_n)\in ON(m)$
                \item da $x_i^{\omega_i}$ nicht in $m$ vorkommt, gilt auch $(\omega_1,\ldots, \overline{\omega_i}, \ldots, \omega_n)\in ON(m)$
                \item aber $(\omega_1,\ldots, \overline{\omega_i}, \ldots, \omega_n)\not\in ON(m')$
                \item \alert{daraus folgt:} $ON(m) \not\subseteq ON(m')$
                \item daher gilt die Behauptung $\square$
              \end{itemize}
            \end{itemize}
          \end{minipage}
        }
      \end{column}
      \begin{column}[t]{0.5\textwidth}
        \resizebox{\textwidth}{!}{
          \begin{minipage}[t]{8cm}
            \begin{itemize}
              \item \alert{Beweis durch Widerspruch:}
              \begin{itemize}
                \item \alert{Annahme:} $ON(m) \subseteq ON(m') \Rightarrow L(m') \subseteq L(m)$ gilt nicht, also $ON(m) \subseteq ON(m') \wedge L(m) \subset L(m')$
                \item \alert{es gilt:} $L(m) \subset L(m')$, wobei $m$ ein Monom ist und $m' = mx_i^{\omega_i}$
                \item man betrachte $(\omega_1,\ldots, \omega_i, \ldots \omega_n)\in ON(m)$
                \item da $x_i^{\omega_i}$ nicht in $m$ vorkommt, gilt auch $(\omega_1,\ldots, \overline{\omega_i}, \ldots, \omega_n)\in ON(m)$
                \item aber $(\omega_1,\ldots, \overline{\omega_i}, \ldots, \omega_n)\not\in ON(m')$
                \item \alert{daraus folgt:} $ON(m') \subset ON(m)$
                \item \alert{Widerspruch}, denn $ON(m) \subset ON(m')$!
                \item die Annahme gilt nicht, also gilt die Behauptung $\square$
              \end{itemize}
            \end{itemize}
          \end{minipage}
        }
      \end{column}
    \end{columns}
    \newpage
\end{frame} 
