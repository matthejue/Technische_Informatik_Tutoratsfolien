%!Tex Root = ../main.tex
% ./Packete.tex
% ./Design.tex
% ./Deklarationen.tex
% ./Vorbereitung.tex
% ./Aufgabe2.tex
% ./Aufgabe3.tex
% ./Aufgabe4.tex
% ./Appendix.tex

\section{Aufgabe 1}

\setcounter{exercise}{1}

\begin{frame}[allowframebreaks]{Aufgabe \thesection}{Implikanten und Primimplikanten}
  \begin{exercisenoinc}
    \ctikzfig{1empty}
  \end{exercisenoinc}
  \begin{exercisenoinc}
    \ctikzfig{1fun}
  \end{exercisenoinc}
  \begin{solution}
      \begin{itemize}
        \item Die konstante 1-Funktion ist kein Implikant, es gibt Belegungen $a \in \mathbb{B}^3$ für die $f(a)=0$
      \end{itemize}
      \resizebox{\textwidth}{!}{
        \begin{minipage}[t]{18cm}
          \ctikzfig{1a}
        \end{minipage}
      }
    \end{solution}
    \begin{solution}
      \begin{itemize}
        \item $a$ ist kein Implikant, also auch kein Primimplikant. Es gilt nicht, dass $\psi(a)\leq f$, denn $\psi(a) (1,0,1)=1\neq f(1,0,1)$
      \end{itemize}
      \resizebox{\textwidth}{!}{
        \begin{minipage}[t]{18cm}
          \ctikzfig{1b}
        \end{minipage}
      }
  \end{solution}
  \begin{solution}
      \begin{itemize}
          \item $a\cdot \overline{b}\cdot \overline{c}$ ist ein Implikant von $f$ , denn $\psi(a\cdot \overline{b}\cdot \overline{c}) \leq f$
          $a\cdot \overline{b}\cdot \overline{c}$ ist aber kein Primimplikant, denn es existiert ein \enquote{größerer} Implikant $a\cdot \overline{c}$ $(\psi(a\cdot \overline{b}\cdot \overline{c}) < \psi(a\cdot \overline{c}))$
      \end{itemize}
      \resizebox{\textwidth}{!}{
        \begin{minipage}[t]{18cm}
          \ctikzfig{1c}
        \end{minipage}
      }
  \end{solution}
  \begin{solution}
      \begin{itemize}
          \item $b\cdot c$ ist ein Primimplikant (und damit natürlich auch ein Implikant). $b\cdot c$ ist maximal, denn es kann kein Literal gestrichen werden, so dass wieder ein Implikant entsteht (weder $b$ noch $c$ sind Implikanten von $f$)
      \end{itemize}
      \resizebox{\textwidth}{!}{
        \begin{minipage}[t]{18cm}
          \ctikzfig{1d}
        \end{minipage}
      }
  \end{solution}
\end{frame}
